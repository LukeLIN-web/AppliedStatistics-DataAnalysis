% Options for packages loaded elsewhere
\PassOptionsToPackage{unicode}{hyperref}
\PassOptionsToPackage{hyphens}{url}
%
\documentclass[
]{article}
\usepackage{amsmath,amssymb}
\usepackage{lmodern}
\usepackage{iftex}
\ifPDFTeX
  \usepackage[T1]{fontenc}
  \usepackage[utf8]{inputenc}
  \usepackage{textcomp} % provide euro and other symbols
\else % if luatex or xetex
  \usepackage{unicode-math}
  \defaultfontfeatures{Scale=MatchLowercase}
  \defaultfontfeatures[\rmfamily]{Ligatures=TeX,Scale=1}
\fi
% Use upquote if available, for straight quotes in verbatim environments
\IfFileExists{upquote.sty}{\usepackage{upquote}}{}
\IfFileExists{microtype.sty}{% use microtype if available
  \usepackage[]{microtype}
  \UseMicrotypeSet[protrusion]{basicmath} % disable protrusion for tt fonts
}{}
\makeatletter
\@ifundefined{KOMAClassName}{% if non-KOMA class
  \IfFileExists{parskip.sty}{%
    \usepackage{parskip}
  }{% else
    \setlength{\parindent}{0pt}
    \setlength{\parskip}{6pt plus 2pt minus 1pt}}
}{% if KOMA class
  \KOMAoptions{parskip=half}}
\makeatother
\usepackage{xcolor}
\usepackage[margin=1in]{geometry}
\usepackage{color}
\usepackage{fancyvrb}
\newcommand{\VerbBar}{|}
\newcommand{\VERB}{\Verb[commandchars=\\\{\}]}
\DefineVerbatimEnvironment{Highlighting}{Verbatim}{commandchars=\\\{\}}
% Add ',fontsize=\small' for more characters per line
\usepackage{framed}
\definecolor{shadecolor}{RGB}{248,248,248}
\newenvironment{Shaded}{\begin{snugshade}}{\end{snugshade}}
\newcommand{\AlertTok}[1]{\textcolor[rgb]{0.94,0.16,0.16}{#1}}
\newcommand{\AnnotationTok}[1]{\textcolor[rgb]{0.56,0.35,0.01}{\textbf{\textit{#1}}}}
\newcommand{\AttributeTok}[1]{\textcolor[rgb]{0.77,0.63,0.00}{#1}}
\newcommand{\BaseNTok}[1]{\textcolor[rgb]{0.00,0.00,0.81}{#1}}
\newcommand{\BuiltInTok}[1]{#1}
\newcommand{\CharTok}[1]{\textcolor[rgb]{0.31,0.60,0.02}{#1}}
\newcommand{\CommentTok}[1]{\textcolor[rgb]{0.56,0.35,0.01}{\textit{#1}}}
\newcommand{\CommentVarTok}[1]{\textcolor[rgb]{0.56,0.35,0.01}{\textbf{\textit{#1}}}}
\newcommand{\ConstantTok}[1]{\textcolor[rgb]{0.00,0.00,0.00}{#1}}
\newcommand{\ControlFlowTok}[1]{\textcolor[rgb]{0.13,0.29,0.53}{\textbf{#1}}}
\newcommand{\DataTypeTok}[1]{\textcolor[rgb]{0.13,0.29,0.53}{#1}}
\newcommand{\DecValTok}[1]{\textcolor[rgb]{0.00,0.00,0.81}{#1}}
\newcommand{\DocumentationTok}[1]{\textcolor[rgb]{0.56,0.35,0.01}{\textbf{\textit{#1}}}}
\newcommand{\ErrorTok}[1]{\textcolor[rgb]{0.64,0.00,0.00}{\textbf{#1}}}
\newcommand{\ExtensionTok}[1]{#1}
\newcommand{\FloatTok}[1]{\textcolor[rgb]{0.00,0.00,0.81}{#1}}
\newcommand{\FunctionTok}[1]{\textcolor[rgb]{0.00,0.00,0.00}{#1}}
\newcommand{\ImportTok}[1]{#1}
\newcommand{\InformationTok}[1]{\textcolor[rgb]{0.56,0.35,0.01}{\textbf{\textit{#1}}}}
\newcommand{\KeywordTok}[1]{\textcolor[rgb]{0.13,0.29,0.53}{\textbf{#1}}}
\newcommand{\NormalTok}[1]{#1}
\newcommand{\OperatorTok}[1]{\textcolor[rgb]{0.81,0.36,0.00}{\textbf{#1}}}
\newcommand{\OtherTok}[1]{\textcolor[rgb]{0.56,0.35,0.01}{#1}}
\newcommand{\PreprocessorTok}[1]{\textcolor[rgb]{0.56,0.35,0.01}{\textit{#1}}}
\newcommand{\RegionMarkerTok}[1]{#1}
\newcommand{\SpecialCharTok}[1]{\textcolor[rgb]{0.00,0.00,0.00}{#1}}
\newcommand{\SpecialStringTok}[1]{\textcolor[rgb]{0.31,0.60,0.02}{#1}}
\newcommand{\StringTok}[1]{\textcolor[rgb]{0.31,0.60,0.02}{#1}}
\newcommand{\VariableTok}[1]{\textcolor[rgb]{0.00,0.00,0.00}{#1}}
\newcommand{\VerbatimStringTok}[1]{\textcolor[rgb]{0.31,0.60,0.02}{#1}}
\newcommand{\WarningTok}[1]{\textcolor[rgb]{0.56,0.35,0.01}{\textbf{\textit{#1}}}}
\usepackage{graphicx}
\makeatletter
\def\maxwidth{\ifdim\Gin@nat@width>\linewidth\linewidth\else\Gin@nat@width\fi}
\def\maxheight{\ifdim\Gin@nat@height>\textheight\textheight\else\Gin@nat@height\fi}
\makeatother
% Scale images if necessary, so that they will not overflow the page
% margins by default, and it is still possible to overwrite the defaults
% using explicit options in \includegraphics[width, height, ...]{}
\setkeys{Gin}{width=\maxwidth,height=\maxheight,keepaspectratio}
% Set default figure placement to htbp
\makeatletter
\def\fps@figure{htbp}
\makeatother
\setlength{\emergencystretch}{3em} % prevent overfull lines
\providecommand{\tightlist}{%
  \setlength{\itemsep}{0pt}\setlength{\parskip}{0pt}}
\setcounter{secnumdepth}{-\maxdimen} % remove section numbering
\ifLuaTeX
  \usepackage{selnolig}  % disable illegal ligatures
\fi
\IfFileExists{bookmark.sty}{\usepackage{bookmark}}{\usepackage{hyperref}}
\IfFileExists{xurl.sty}{\usepackage{xurl}}{} % add URL line breaks if available
\urlstyle{same} % disable monospaced font for URLs
\hypersetup{
  hidelinks,
  pdfcreator={LaTeX via pandoc}}

\title{STAT 210\\
Applied Statistics and Data Analysis:\\
Homework 5}
\author{}
\date{\vspace{-2.5em}Due on Oct.~9/2022}

\begin{document}
\maketitle

\hypertarget{question-1-60-pts}{%
\subsection{Question 1 (60 pts)}\label{question-1-60-pts}}

For this question, use again the data set \texttt{human} that we used in
HW2. Read the file \texttt{Human\_data.txt} and store this in an object
called human.

\begin{enumerate}
\def\labelenumi{(\alph{enumi})}
\item
  The body mass index (BMI) is defined as a person's weight in kilograms
  divided by the square of height in meters. Add a column named
  \texttt{bmi} to the data frame with the value of this index for each
  subject.
\item
  Using the function \texttt{cut} create a new variable \texttt{bmi.fac}
  in \texttt{human} by dividing the subjects into four categories
  according to the value of \texttt{bmi}: below 20 corresponds to
  underweight, greater than 20 and up to 25 is normal, greater than 25
  and up to 30 is overweight and above 30 is obese.
\item
  Build a contingency table of \texttt{Gender} and the factor you
  created in (b). \texttt{Gender} should correspond to the rows of your
  table.
\item
  Do a mosaic plot for the table in (c). Comment on what you observe on
  this graph.
\item
  Add a margin row and column to the table in (c) with the corresponding
  totals.
\item
  Build a table with the proportions with respect to the total number of
  cases for each gender. Comment on the results.
\item
  We want to test whether the distribution of the \texttt{bmi}
  categories that you created is the same for the different genders.
  What test would you use for this and why? What conditions need to be
  satisfied? Discuss whether they are in this example. Carry out this
  test and comment on your results.
\end{enumerate}

\begin{Shaded}
\begin{Highlighting}[]
\NormalTok{human }\OtherTok{=} \FunctionTok{read.table}\NormalTok{(}\StringTok{"Human\_data.txt"}\NormalTok{,}\AttributeTok{header =} \ConstantTok{TRUE}\NormalTok{)}
\NormalTok{human }\OtherTok{\textless{}{-}} \FunctionTok{within}\NormalTok{(human, bmi }\OtherTok{\textless{}{-}}\NormalTok{ human}\SpecialCharTok{$}\NormalTok{Weight\_kg}\SpecialCharTok{/}\NormalTok{ (human}\SpecialCharTok{$}\NormalTok{Height\_cm}\SpecialCharTok{/}\DecValTok{100}\NormalTok{)}\SpecialCharTok{\^{}}\DecValTok{2}\NormalTok{)}
\NormalTok{human }\OtherTok{\textless{}{-}} \FunctionTok{within}\NormalTok{(human, bmi.fac }\OtherTok{\textless{}{-}} \FunctionTok{cut}\NormalTok{(human}\SpecialCharTok{$}\NormalTok{bmi, }\FunctionTok{c}\NormalTok{(}\SpecialCharTok{{-}}\ConstantTok{Inf}\NormalTok{,}\DecValTok{20}\NormalTok{,}\DecValTok{25}\NormalTok{,}\DecValTok{30}\NormalTok{,}\ConstantTok{Inf}\NormalTok{),}\AttributeTok{labels =} \FunctionTok{c}\NormalTok{(}\StringTok{"underweight"}\NormalTok{,}\StringTok{"normal"}\NormalTok{,}\StringTok{"overweight"}\NormalTok{,}\StringTok{"obese"}\NormalTok{)))}
\NormalTok{(}\AttributeTok{tab1 =} \FunctionTok{table}\NormalTok{(human}\SpecialCharTok{$}\NormalTok{Gender,human}\SpecialCharTok{$}\NormalTok{bmi.fac))}
\end{Highlighting}
\end{Shaded}

\begin{verbatim}
##    
##     underweight normal overweight obese
##   F          12     80        128    52
##   M           2     48        122    56
\end{verbatim}

\begin{Shaded}
\begin{Highlighting}[]
\FunctionTok{mosaicplot}\NormalTok{(tab1[}\DecValTok{1}\SpecialCharTok{:}\DecValTok{2}\NormalTok{,}\DecValTok{1}\SpecialCharTok{:}\DecValTok{4}\NormalTok{], }\AttributeTok{col =} \FunctionTok{c}\NormalTok{(}\StringTok{\textquotesingle{}dodgerblue\textquotesingle{}}\NormalTok{,}\StringTok{\textquotesingle{}skyblue1\textquotesingle{}}\NormalTok{),}\AttributeTok{main =} \StringTok{\textquotesingle{}Gender with bmi.fac\textquotesingle{}}\NormalTok{)}
\end{Highlighting}
\end{Shaded}

\includegraphics{HW5-2022_files/figure-latex/unnamed-chunk-1-1.pdf}

\begin{Shaded}
\begin{Highlighting}[]
\NormalTok{(human.table }\OtherTok{\textless{}{-}} \FunctionTok{addmargins}\NormalTok{(tab1))}
\end{Highlighting}
\end{Shaded}

\begin{verbatim}
##      
##       underweight normal overweight obese Sum
##   F            12     80        128    52 272
##   M             2     48        122    56 228
##   Sum          14    128        250   108 500
\end{verbatim}

\begin{Shaded}
\begin{Highlighting}[]
\FunctionTok{prop.table}\NormalTok{(tab1)}
\end{Highlighting}
\end{Shaded}

\begin{verbatim}
##    
##     underweight normal overweight obese
##   F       0.024  0.160      0.256 0.104
##   M       0.004  0.096      0.244 0.112
\end{verbatim}

\begin{Shaded}
\begin{Highlighting}[]
\FunctionTok{str}\NormalTok{(tab1)}
\end{Highlighting}
\end{Shaded}

\begin{verbatim}
##  'table' int [1:2, 1:4] 12 2 80 48 128 122 52 56
##  - attr(*, "dimnames")=List of 2
##   ..$ : chr [1:2] "F" "M"
##   ..$ : chr [1:4] "underweight" "normal" "overweight" "obese"
\end{verbatim}

\begin{Shaded}
\begin{Highlighting}[]
\FunctionTok{chisq.test}\NormalTok{(tab1)}
\end{Highlighting}
\end{Shaded}

\begin{verbatim}
## 
##  Pearson's Chi-squared test
## 
## data:  tab1
## X-squared = 11.653, df = 3, p-value = 0.00867
\end{verbatim}

I observe that there are more obese males than females and fewer
underweight males than females on this graph. I observe that overweight
people have most portion no matter females or males.

I use Chi-Squared Test. Because the chi-square test is a non-parametric
test, and the non-parametric test does not have the assumptions of
specific parameters and overall normal distribution. Condition : p
\textless{} 0.01, so that bmi is not the same for different genders. BMI
has a relation with the genders.

\hypertarget{question-2-40-pts}{%
\subsection{Question 2 (40 pts)}\label{question-2-40-pts}}

Are newborn babies more likely to be boys than girls?

\begin{enumerate}
\def\labelenumi{(\alph{enumi})}
\tightlist
\item
  In the city of Comala, out of 5235 babies born, 2705 were boys. Is
  this evidence that boys are more common than girls? State clearly the
  statistical procedure that you are using to answer this question.
  Describe the assumptions that you make. Are they reasonable in this
  case? Carry out this procedure and discuss your results.
\end{enumerate}

\begin{Shaded}
\begin{Highlighting}[]
\NormalTok{tmp }\OtherTok{=} \FunctionTok{c}\NormalTok{(}\FunctionTok{rep}\NormalTok{(}\FunctionTok{c}\NormalTok{(}\StringTok{\textquotesingle{}B\textquotesingle{}}\NormalTok{), }\AttributeTok{times=}\DecValTok{2705}\NormalTok{) , }\FunctionTok{rep}\NormalTok{(}\FunctionTok{c}\NormalTok{(}\StringTok{\textquotesingle{}G\textquotesingle{}}\NormalTok{), }\AttributeTok{times=}\DecValTok{5235{-}2705}\NormalTok{) )}
\NormalTok{df }\OtherTok{\textless{}{-}} \FunctionTok{data.frame}\NormalTok{(}\AttributeTok{id=}\FunctionTok{c}\NormalTok{(}\DecValTok{1}\SpecialCharTok{:}\DecValTok{5235}\NormalTok{),}\AttributeTok{sex=}\NormalTok{tmp)}
\NormalTok{(}\AttributeTok{prfs =} \FunctionTok{xtabs}\NormalTok{( }\SpecialCharTok{\textasciitilde{}}\NormalTok{ sex, }\AttributeTok{data=}\NormalTok{df))}
\end{Highlighting}
\end{Shaded}

\begin{verbatim}
## sex
##    B    G 
## 2705 2530
\end{verbatim}

\begin{Shaded}
\begin{Highlighting}[]
\FunctionTok{str}\NormalTok{(prfs)}
\end{Highlighting}
\end{Shaded}

\begin{verbatim}
##  'xtabs' int [1:2(1d)] 2705 2530
##  - attr(*, "dimnames")=List of 1
##   ..$ sex: chr [1:2] "B" "G"
##  - attr(*, "call")= language xtabs(formula = ~sex, data = df)
\end{verbatim}

\begin{Shaded}
\begin{Highlighting}[]
\FunctionTok{chisq.test}\NormalTok{(prfs)}
\end{Highlighting}
\end{Shaded}

\begin{verbatim}
## 
##  Chi-squared test for given probabilities
## 
## data:  prfs
## X-squared = 5.85, df = 1, p-value = 0.01558
\end{verbatim}

\begin{Shaded}
\begin{Highlighting}[]
\CommentTok{\#tab \textless{}{-} matrix(c(7, 5, 14, 19, 3, 2, 17, 6, 12), ncol=3, byrow=TRUE)}
\end{Highlighting}
\end{Shaded}

We use Chi-Squared Test.

H0: In the general, newborn babies are not more likely to be boys than
girls. H1 : In the general, newborn babies are more likely to be boys
than girls. p \textgreater0.01,accept H0,newborn babies are not more
likely to be boys than girls. if we choose p\textless0.05, newborn
babies are more likely to be boys than girls.

\begin{enumerate}
\def\labelenumi{(\alph{enumi})}
\setcounter{enumi}{1}
\tightlist
\item
  In the city of Macondo, out of 3765 babies born, 1905 were boys. Is
  there evidence that the frequency of boys is different in these two
  cities? Again, state clearly the statistical procedure that you are
  using to answer this question. Describe the assumptions that you make.
  Are they reasonable in this case? Carry out this procedure and discuss
  your results.
\end{enumerate}

\begin{Shaded}
\begin{Highlighting}[]
\NormalTok{tab }\OtherTok{=} \FunctionTok{matrix}\NormalTok{(}\FunctionTok{c}\NormalTok{(}\DecValTok{2705}\NormalTok{, }\DecValTok{5235{-}2705}\NormalTok{,}\DecValTok{1905}\NormalTok{,}\DecValTok{3765{-}1905}\NormalTok{),}\AttributeTok{ncol=}\DecValTok{2}\NormalTok{,}\AttributeTok{byrow=}\ConstantTok{TRUE}\NormalTok{)}
\CommentTok{\#boy = c(2705, 1905)}
\CommentTok{\#girl = c(5235{-}2705,3765{-}1905)}
\CommentTok{\#label = c("boys","grils")}
\FunctionTok{colnames}\NormalTok{(tab) }\OtherTok{=} \FunctionTok{c}\NormalTok{(}\StringTok{\textquotesingle{}boy\textquotesingle{}}\NormalTok{,}\StringTok{\textquotesingle{}girl\textquotesingle{}}\NormalTok{)}
\FunctionTok{rownames}\NormalTok{(tab) }\OtherTok{=} \FunctionTok{c}\NormalTok{(}\StringTok{\textquotesingle{}Comala\textquotesingle{}}\NormalTok{,}\StringTok{\textquotesingle{}Macondo\textquotesingle{}}\NormalTok{)}
\CommentTok{\#df \textless{}{-} data.frame(labels= label, boys=boy,girls=girl  )}
\NormalTok{prfs }\OtherTok{=} \FunctionTok{as.table}\NormalTok{(tab)}
\FunctionTok{str}\NormalTok{(prfs)}
\end{Highlighting}
\end{Shaded}

\begin{verbatim}
##  'table' num [1:2, 1:2] 2705 1905 2530 1860
##  - attr(*, "dimnames")=List of 2
##   ..$ : chr [1:2] "Comala" "Macondo"
##   ..$ : chr [1:2] "boy" "girl"
\end{verbatim}

\begin{Shaded}
\begin{Highlighting}[]
\FunctionTok{print}\NormalTok{(prfs)}
\end{Highlighting}
\end{Shaded}

\begin{verbatim}
##          boy girl
## Comala  2705 2530
## Macondo 1905 1860
\end{verbatim}

\begin{Shaded}
\begin{Highlighting}[]
\FunctionTok{chisq.test}\NormalTok{(prfs)}
\end{Highlighting}
\end{Shaded}

\begin{verbatim}
## 
##  Pearson's Chi-squared test with Yates' continuity correction
## 
## data:  prfs
## X-squared = 0.9682, df = 1, p-value = 0.3251
\end{verbatim}

The χ2 test can also be used to test for independence of categorical
variables in contingency tables. We use table to produce the contingency
table for these two variables.

We use Chi-Squared Test.

H0: there is no evidence that the frequency of boys is different in
these two cities. H1 : there is evidence that the frequency of boys is
different in these two cities.

p\textgreater0.05, we accept H0, there is no evidence that the frequency
of boys is different in these two cities.

\end{document}
